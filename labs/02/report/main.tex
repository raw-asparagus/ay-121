\documentclass[12pt,preprint]{aastex} 
\usepackage{amsmath, amsfonts, amssymb}
\usepackage{hyperref}

\begin{document}

\title{AY121 Lab 2: 21-cm Hydrogen Line Observations}

\author{Jun Rui Ting \\ \today}

\begin{abstract}

\end{abstract}

%%%%%%%%%%%%%%%%%%%%%%%%%%%%%%%%%%%%%%%%%%%%%%%%%%%%%%%%%%%%%%%%%%%%%%%%%%%%%%%%
%       Introduction
%%%%%%%%%%%%%%%%%%%%%%%%%%%%%%%%%%%%%%%%%%%%%%%%%%%%%%%%%%%%%%%%%%%%%%%%%%%%%%%%

\section{Introduction}
\label{introduction}

\section{Theory}
\label{theory}

\subsection{Time measures}

Discuss different time keeping systems and also digital timekeeping and
synchronization.

\subsection{Coordinate systems and measures}

Introduce the different coordinate systems.

Why rotation matrices are a convenient choice.

\subsection{The 21-cm hydrogen line}

\begin{itemize}

\item Hyperfine emission from neutral hydrogen

\end{itemize}

\subsubsection{Doppler shifting}

The Milky Way is rotating and different arms of the Milky Way are moving at
different velocities with respect to us, which leads to different Doppler shifts
in the 21-cm line.

\subsection{The radio dish}

\subsubsection{Antenna}

\subsection{Signal propagation through transmission lines}

\subsubsection{The transmission cable}

\subsubsection{Signal propagation and degradation}

Introduce castingt the signal as a $\exp\left(-i\omega t\right)$ wave, then
discuss how the impendance (somewhere in the equation) determines how the signal
degrades as it propagates through the cable.

\subsubsection{Signal reflections}

Same as above but on impedance and reflections.

%%%%%%%%%%%%%%%%%%%%%%%%%%%%%%%%%%%%%%%%%%%%%%%%%%%%%%%%%%%%%%%%%%%%%%%%%%%%%%%%
%       Materials and Methods
%%%%%%%%%%%%%%%%%%%%%%%%%%%%%%%%%%%%%%%%%%%%%%%%%%%%%%%%%%%%%%%%%%%%%%%%%%%%%%%%

\section{Materials and Methods}
\label{materials_and_methods}

\subsection{Equipment}

\begin{itemize}

\item RTL832U Software Defined Radio

\item Keysight N9310A RF Signal Generator DS345

\item (whatever cables we used)

\item Horn antenna on New Campbell Hall

\end{itemize}

\subsection{Software engineering}

\subsection{Data Collection Pipeline}

Include a flowchart of the data collection pipeline, and discuss how the data is
collected, stored, and processed.

Include a schema of the data storage, and how the data is processed to get the
final results.

\subsection{Physical calibration}

Discuss how horizontal (topological) coordinates can be elucidated by the
altitude notches on the horn antenna, and how the azimuthal coordinates can be
elucidated by using a compass. 

Discuss the Rayleigh criterion of the horn antenna and the resolution of the
antenna.

Using the two above discussions, discuss the tolerances of the measurements,
mentioning drifts and ideal observation windows for the same patch of the sky.

\subsection{Signal calibration pipeline}

\begin{enumerate}

\item Refer to \texttt{cal\_intensity.pdf}.

\item Using a smoothing kernel/convolution to smooth the power spectrum density of the signal:
\begin{enumerate}

\item Rectangular

\item Gaussian

\item Savitzky-Golay

\end{enumerate}

\end{enumerate}

\section{Experiments}
\label{experiments}

\subsection{Experiment 1: Finding out what is a good \texttt{nblocks} to use}

\begin{itemize}

\item Point at North Galactic Pole to compare noise levels (via radiometer
equation and other methods).

\end{itemize}

\subsection{Experiment 2: Calibrating the power of the signal (find out a scale
for the power spectrum density of the signal)}

%%%%%%%%%%%%%%%%%%%%%%%%%%%%%%%%%%%%%%%%%%%%%%%%%%%%%%%%%%%%%%%%%%%%%%%%%%%%%%%%
%       Results
%%%%%%%%%%%%%%%%%%%%%%%%%%%%%%%%%%%%%%%%%%%%%%%%%%%%%%%%%%%%%%%%%%%%%%%%%%%%%%%%

\section{Results}
\label{results}

Ensure all results are reproted with error bars.

Create simualted phenomenological models to compare each experiment with.

%%%%%%%%%%%%%%%%%%%%%%%%%%%%%%%%%%%%%%%%%%%%%%%%%%%%%%%%%%%%%%%%%%%%%%%%%%%%%%%%
%       Discussion
%%%%%%%%%%%%%%%%%%%%%%%%%%%%%%%%%%%%%%%%%%%%%%%%%%%%%%%%%%%%%%%%%%%%%%%%%%%%%%%%

\section{Discussion}
\label{discussion}

\subsubsection{How do we know what we are seeing is the spiral arms of the Milky
Way?}

%%%%%%%%%%%%%%%%%%%%%%%%%%%%%%%%%%%%%%%%%%%%%%%%%%%%%%%%%%%%%%%%%%%%%%%%%%%%%%%%
%       Conclusion
%%%%%%%%%%%%%%%%%%%%%%%%%%%%%%%%%%%%%%%%%%%%%%%%%%%%%%%%%%%%%%%%%%%%%%%%%%%%%%%%

\section{Conclusion}
\label{conclusion}

\section*{Acknowledgements}

Special thanks to my lab partners Nathaneil and Kayhan for their help in this
Lab, and Aaron and Ben for their guidance and support throughout the Lab. Also,
I would like to acknowledge the use of \texttt{numpy}, \dots

\citep{gaia_dr2_2018}

\bibliography{ref}
\bibliographystyle{aasjournal}

\end{document}
